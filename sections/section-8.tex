\section{Some Notes On Development Of A Structual Stress With Time \\随着时间发展结构强度的一些注意事项}

\begin{paracol}{2}
    
    Because the samples showed initially the same structure and because the secondary consolidation was insignificant, the samples for each mode of consolidation had the same structure prior to the shear test. At a given strain it is believed that approximately the same number of contact points are broken, resulting in identical magnitudes of pore pressure set up due to transference to the pore-water of effective stresses originally carried at the contact points. To produce a given strain, a higher value of $\sigma_1'/\sigma_3'$is required for the samples consolidated 4 months than for those consolidated only 3 days. The conceivable effect of aging is therefore a growth of bond strength at the points of contact of the particles.

    \switchcolumn

    由于样品最初显示出相同的结构,并且由于次固结不明显,因此在剪切试验之前,每种固结模式的样品都具有相同的结构。 在给定的应变下,据信破裂了大约相同数量的接触点,这导致了相同大小的孔隙压力,这是由于最初在接触点处承载的有效应力转移到孔隙水中而形成的。 为了产生给定的应变,固结4个月的样品比仅固结3天的样品需要更高的$\sigma_1'/\sigma_3'$值。 因此,老化的可能效果是在颗粒接触点处粘结强度的增加。

    \switchcolumn*

    As the strain increases, movement occurs at most or all of the contact points and the bonds developed during the consolidation period are destroyed. The $\sigma_1'/\sigma_3'$ values for different aging periods therefore gradually converge into a single curve and their final maximum value is independent of the initial bond strength. The additional cohesive component of the shear strength gained with time during a consolidation period is therefore destroyed, at small strains and does not contribute to the final strength of the soil skeleton.

    \switchcolumn
    
    随着应变增加,运动最多或所有接触点发生,固结期间形成的键被破坏。因此,不同老化时期的$\sigma_1'/\sigma_3'$值逐渐收敛为一条曲线,其最终最大值与初始粘结强度无关。因此,在固结期间随时间获得的抗剪强度的附加内聚成分会在小应变下被破坏,并且不会增加土壤骨架的最终强度。

    \switchcolumn*

    It is reasonable to expect that the structure of the soil samples varies with different modes of consolidation. It is therefore also conceivable that the pore-pressure and effective stressratio characteristics might be slightly different during the early stages of the shearing process of the isotropically and the anisotropically consolidated samples. At large strains when the initial bonds have been destroyed, the mechanical behaviour of the particles is gradually dominated by sliding. It is therefore logical to expect that the ultimate values of the effective principal stress ratio would be the same for both isotropically and anisotropically consolidated samples, and this has been shown to be the case, as seen from \autoref{table:3}.

    \switchcolumn
    
    可以合理预期土壤样品的结构会随着固结模式的不同而变化。因此,还可以想象,在各向同性和各向异性固结样品的剪切过程的早期,孔隙压力和有效应力特性可能会略有不同。在较大的应变下,当初始键被破坏时,颗粒的机械行为逐渐受到滑动的支配。因此,可以合理地预期,各向同性和各向异性固结样品的有效主应力比的最终值将相同,并且从\cntableref{table:3}中可以看出情况确实如此。

    \switchcolumn*
    
    There are no reasons to believe that the bonds developed at the contact points during the 4 months of aging included in the present test programme represent the maximum values which can be reached in geological time. On the contrary, there are good reasons to believe that these bonds, which do not show up in conventional shear tests on reconsolidated samples, are of considerably greater importance in nature than observed in the tests.

    \switchcolumn
   
    没有理由相信本试验程序中包括的在老化的4个月期间在接触点处形成的键代表了在地质时间内可以达到的最大值。相反,有充分的理由相信,这些结合力在自然界比在试验中观察到的重要性要大得多,这些结合力在常规的重塑样品的剪切试验中没有表现出来。

    \switchcolumn*
    
    Even if the test results do not permit a quantitative evaluation of the effect of aging over periods of time on a geological scale, they indicate clearly in a qualitative way the general trends of further aging. A continued growth of bonds at the contact points will lead to a further increase of the steepness of the early part of the stress-strain curve of the clay and the undrained shear strength will therefore be mobilized at a smaller strain. Assuming that the aging has no effect on the pore-pressure/strain relationship as observed in the tests, the excess pore pressure at an undrained failure will be reduced. The degree of mobilization of the effective stress shear strength parameters will therefore also be reduced.

    \switchcolumn
   
    即使试验结果不允许定量评估一段时间内的老化对地质规模的影响,它们也以定性方式清楚地表明了进一步老化的总体趋势。接触点处粘结的持续增长将导致黏土应力-应变曲线早期部分的陡度进一步增加,因此不排水的剪切强度将在较小的应变下动员。如试验中所观察到的那样,假设老化对孔压/应变关系没有影响,则不排水破坏时的过大孔压将减少。因此,有效应力剪切强度参数的动员程度也将降低。

    \switchcolumn*
    
    With increasing consolidation time the undrained shear strength will thus to an increasing degree be controlled by the contribution to the strength of the cohesive bonds prevailing at small strains. The frictional component, the mobilization of which requires strain, will be reduced correspondingly.

    \switchcolumn
   
    随着固结时间的增加,不排水的剪切强度将因此受到小应变时对粘结力强度的贡献的控制。相应地减少了其运动需要应变的摩擦分量。

    \switchcolumn*
    
    It should finally be mentioned that the existence of cohesive bonds in clays which are developed with time and are destroyed completely or partly by reconsolidation has previously been reported for a clay from the Gota valley (Bjerrum and Wu, 1960). Terzaghi (1941) suggested that such bonds may be the controlling factor for the equilibrium water contents of natural clay sediments and recent experiments carried out at Purdue University (Leonard and Ramiah, 1960) have demonstrated by consolidation tests on aged samples that cohesive bonds were formed with time leading to an increased resistance against volume reduction.

    \switchcolumn
   
    最后应该提到的是,以前已经报道了来自哥达河谷的黏土中黏合键的存在,黏黏键会随着时间的发展而被完全或部分破坏\citep{Bjerrum1960109}。\citet{Terzaghi1941211}认为这样的键可能是天然黏土沉积物平衡水含量的控制因素,最近在普渡大学进行的实验\citep{Leonards1960116}通过对老化样品的固结试验证明形成了内聚键随着时间的流逝,导致体积减小的阻力增加。

\end{paracol}