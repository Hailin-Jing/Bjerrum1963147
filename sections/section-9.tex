\section{Conclusion 结论}

\begin{paracol}{2}
    
    In the described test series, normally consolidation samples were aged for different periods of time before they were subjected to undrained shear tests in the triaxial apparatus. The results can be summarized as follows:

    \switchcolumn

    在所描述的试验系列中,正常固结样品在三轴设备中经受不排水的剪切试验之前,先老化了不同的时间。 结果总结如下:

    \switchcolumn*

    \begin{enumerate}
        \item Samples with longer periods of consolidation show at early stages of the tests a greater resistance against a shear distortion. The effective principal stress ratio increases more rapidly at small strains for the older samples. For larger strains this effect of aging diminishes and it disappears completely when the ultimate value of the effective principal stress ratio is reached.
        \item The excess pore-pressure/strain relationship observed during the shear tests were independent of the age of the samples.
        \item The principal stress difference reaches its maximum value at a smaller strain the longer the sample has been left for aging. The maximum value shows a slight increase with the age of the samples.
        \item A study of the stress-strain and pore-pressure/strain curves suggests that the change of behaviour of samples for increasing age is a result of the bonds which develop at contact points between the clay particles. These bonds are gradually destroyed for increasing shear strain. The cohesive contribution to the undrained shear strength is thus believed to increase and the frictional contribution to decrease with the age of the sample.
    \end{enumerate}

    \switchcolumn
    
    \begin{enumerate}
        \item 固结时间较长的样品在试验的早期阶段显示出对剪切变形的更大抵抗力。 对于较旧的样品,有效主应力比在较小应变下会更快地增加。 对于较大的应变,当达到有效主应力比的最终值时,老化的影响就会减弱,并且完全消失。
        \item 在剪切试验中观察到的多余的孔压/应变关系与样品的年龄无关。
        \item 样品保持老化的时间越长,在较小的应变下主应力差达到最大值。 最大值显示随着样品寿命的增加而略有增加。
        \item 对应力-应变和孔隙压力/应变曲线的研究表明,随着年龄的增长,样品行为的变化是由于在黏土颗粒之间的接触点形成的键的结果。 这些键逐渐被破坏以增加剪切应变。 因此,随着样品的老化,对不排水剪切强度的内聚作用增加,而摩擦作用减少。
    \end{enumerate}

\end{paracol}