\section{Introduction 介绍}

\begin{paracol}{2}

    As a part of a current research on the stress-strain properties of normally consolidated clays an investigation has been made on the influence of the time on the undrained shearstrength and pore-pressure characteristics of a normally consolidated clay.

    \switchcolumn

    作为对正常固结黏土的应力-应变特性的当前研究的一部分,已经研究了时间对正常固结黏土的不排水剪切强度和孔隙压力特性的影响。

    \switchcolumn*

    There are two different ways in which the time factor may influence the results obtained by a conventional triaxial shear test. In the first place it is a well-known fact that the rate of application of the shear stresses has an influence on the shear strength of certain clay and shale \citep{Casacrande1951}.

    \switchcolumn
        
    时间因素可以通过两种不同的方式影响通过常规三轴剪切试验获得的结果。 首先,众所周知的事实是,剪切应力的施加速率会影响某些黏土和页岩的剪切强度\citep{Casacrande1951}。

    \switchcolumn*
    
    Recent research \citep{Bjerrum1958148} has, however, shown that in the second place a clay will behave differently depending on how long a time it has been left in the triaxial cell for consolidation. This means that samples consolidated for different periods of time prior to the shear testing might behave differently during a subsequent shear test depending on the "age" of the sample.

    \switchcolumn
            
    然而,最近的研究\citep{Bjerrum1958148}表明,第二方面,黏土的行为会因将其留在三轴仪中进行固结的时间而有所不同。 这意味着在剪切试验之前经过不同时间固结的样品在随后的剪切试验中可能会根据样品的“年龄”而表现不同。

    \switchcolumn*

    In the above-mentioned Paper it was, for instance, shown that clays which exhibit secondary consolidation will gain in shear strength with time. But also clays which do not show a secondary time effect might change their properties with time and the tests described in the Paper indicated that with time a clay might become more brittle, showing lower strains at failure.

    \switchcolumn
        
    例如,在上述论文中,显示出二次固结的黏土的抗剪强度会随时间增加。 但是,未表现出二次时间效应的黏土也可能随时间改变其性能,并且本文所述的试验表明,随着时间的推移,黏土可能会变得更脆,显示出较低的破坏应变。

    \switchcolumn*

    This last-mentioned effect of time is of special interest for an evaluation of the undrained shear strength of normally consolidated clays. As pointed out in various Papers \citep{Bjerrum1960711, Bjerrum196123a}, there is a serious disagreement between the undrained shear strengths measured on undisturbed clay and on samples which have been reconsolidated in the laboratory. A possible explanation of this discrepancy may eventually be that an undisturbed clay since its deposition has gained some properties which cannot be reproduced in the laboratory where the time for which the sample is consolidated is of the order of a few days only. A reference should here be given to a previous study of this problem, made by \citet{Taylor195363}.

    \switchcolumn
        
    最后提到的时间影响对于评估正常固结的黏土的不排水剪切强度特别有意义。 正如各种论文所指出的那样\citep{Bjerrum1960711, Bjerrum196123a},在未经扰动的黏土上测量的不排水剪切强度与在实验室中重新固结的样品之间存在着严重的分歧。 这种差异的最终解释可能是未受干扰的黏土,因为其沉积获得了某些特性,而在实验室中,样品固结的时间只有几天左右,这些特性无法在实验室中再现。 这里应该参考以前由\citet{Taylor195363}进行的研究。

    \switchcolumn*

    With the purpose of investigating this factor a series of triaxial tests were performed in which samples of a normally consolidated clay, which did not show an appreciable secondary time effect, were sheared after different periods of aging. It is the results of this senes of tests which will be described below.

    \switchcolumn

    为了研究该因素,进行了一系列三轴试验,其中在不同的老化时间后剪切了未显示明显的二次时间效应的正常固结黏土样品。 这些感觉的结果将在下面描述。
    
\end{paracol}