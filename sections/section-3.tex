\section{Testing Procedure 试验程序}

\begin{paracol}{2}
    
    The tests were carried out with the triaxial equipment developed at the Norwegian Geotechnical Institute. The samples had a diameter of 3.56 cm (10 sq. cm area) and their height was 8 cm. In all tests double membranes separated with a thin layer of silicon grease were used.

    \switchcolumn

    试验是使用挪威岩土工程学院开发的三轴设备进行的。 样品的直径为3.56厘米(面积为10平方厘米),高度为8厘米。 在所有试验中,均使用用硅脂薄层隔开的双膜。

    \switchcolumn*

    In order to ensure that the samples tested were normally consolidated, the consolidation pressure used was 2.5 $\rm{kg/cm^2}$. Six samples were consolidated isotropically at this pressure and six samples were consolidated anisotropically at the major and minor principal stresses of 2.5 $\rm{kg/cm^2}$ and 1.5 $\rm{kg/cm^2}$, respectively. For each series two samples were left at the consolidation pressure for a period of 3 days, two samples for 2 weeks and the last two samples for a period of 4 months.
        
    \switchcolumn
    
    为了确保试验的样品正常固结,固结压力为2.5$\rm{kg/cm^2}$。 在该压力下六个样品进行了各向同性固结,在主应力和次主应力为2.5$\rm{kg/cm^2}$和1.5$\rm{kg/cm^2}$时,六个样品进行了各向异性固结。对于每个系列,将两个样品在固结压力下放置3天,两个样品放置2周,最后两个样品放置4个月。

    \switchcolumn*

    After these periods of aging the samples were subjected to a conventional undrained shear test with measurements of pore pressures. The shear tests were carried out with a back pressure of 2 $\rm{kg/cm^2}$ to ensure saturation, the cell pressure and the pore pressure being raised simultaneously. The strain rate used for the shear test was 1$\%$ in 48 min.

    \switchcolumn

    在进行了这些时间的固结之后,对样品进行常规的不排水剪切试验,其测量孔隙压力。 剪切试验是在2$\rm{kg/cm^2}$的背压下进行的,以确保饱和,同时提高毛细压力和孔隙压力。 用于剪切试验的应变率在48分钟内为1$\%$。

\end{paracol}