\section{Effect Of Aging On $\dfrac{\sigma_1-\sigma_3}{p}-\varepsilon$ Relationship \\时效对$\dfrac{\sigma_1-\sigma_3}{p}-\varepsilon$关系的影响}

\begin{paracol}{2}
    
    From the graphical plots of the test results in \autoref{figure:2} and \autoref{figure:3} it is directly seen that an aging has a very marked influence on the $(\sigma_1-\sigma_3)$ — strain relationship. For an increased period of aging the maximum value of $\sigma_1-\sigma_3$ occurs at a smaller strain and the reduction in $\sigma_1-\sigma_3$ after failure becomes more pronounced.

    \switchcolumn

    从\cnfigureref{figure:2}和\cnfigureref{figure:3}的试验结果的图形图中可以直接看出,时效对$(\sigma_1-\sigma_3)$-应变关系具有非常显着的影响。 随着老化时间的延长,$\sigma_1-\sigma_3$的最大值出现在较小的应变下,而失效后$\sigma_1-\sigma_3$的减小变得更加明显。

    \switchcolumn*

    Considering the maximum values of $\sigma_1-\sigma_3$ there is a slight increase in the undrained compressive strength of the samples with aging. This increase is of the order of $6-10\%$ for the range of time variations investigated in the present test series.

    \switchcolumn
       
    考虑到$\sigma_1-\sigma_3$的最大值,样品的不排水抗压强度会随着老化而略有增加。 对于本试验系列中研究的时间变化范围,这种增加约为$6-10\%$。

    \switchcolumn*

    Now, the value of a1 $(\sigma_1-\sigma_3)/p$ is at any strain a function of the two basic parameters $\Delta{u}/p$ and $\sigma_1'/\sigma_3'$ as expressed by the equation:

    \switchcolumn
       
    现在,在任何应变下,$(\sigma_1-\sigma_3)/p$的值都是两个基本参数$\Delta{u}/p$和$\sigma_1'/\sigma_3'$的函数,如方程式所示: 

\end{paracol}

\begin{align}
    \dfrac{\sigma_1-\sigma_3}{p}=(\sigma_1'/\sigma_3'-1)\left\{1-\left[\frac{\Delta{u}}{p}+(1-K)\right]\right\}
\end{align}

\begin{paracol}{2}

    $(\sigma_1-\sigma_3)/p$ is thus a product of a "strength term" $(\sigma_1'/\sigma_3'-1)$ which increases with strain and an "effective stress term" $ \left\{1-\left[\frac{\Delta{u}}{p}+(1-K)\right]\right\}$ which in undramed tests on normally consolidated clays decreases for increasing strain. The maximum value of $(\sigma_1-\sigma_3)/p$ will, therefore, in general not occur at the strain where the stren?th term is maximum, but at an intermediate strain where the rate of decrease of the effective stress term just compensates the rate of increase of the strength term, i.e. when:

    \switchcolumn

    因此,$(\sigma_1-\sigma_3)/p$是“强度项”$(\sigma_1'/\sigma_3'-1)$的乘积项,该强度项随应变而增加,而“有效应力项”$ \left\{1-\left[\frac{\Delta{u}}{p}+(1-K)\right]\right\}$在正常固结的黏土上,随应变的增加而减少。 因此,$(\sigma_1-\sigma_3)/p$的最大值通常不会出现在强度项最大的应变处,而出现在中间应变处,有效应力项的减小率只会补偿强度的增加率,例如,当下述条件满足时:

\end{paracol}

\begin{align}
    \frac{\dfrac{d\left(\dfrac{\Delta{u}}{p}\right)}{d\varepsilon}}{1-\left[\dfrac{\Delta{u}}{p}+(1-K)\right]}=\frac{\dfrac{d(\sigma_1'/\sigma_3')}{d\varepsilon}}{\sigma_1'/\sigma_3'-1}
\end{align}

\begin{paracol}{2}

    This fact explains the observation from tests on normally consolidated clays that there is a rise in pore pressure at the peak value of $\sigma_1-\sigma_3$.

    \switchcolumn

    这一事实解释了从对正常固结黏土的试验中观察到的结果,即在$\sigma_1-\sigma_3$的峰值处,孔隙压力有所增加。

    \switchcolumn*

    Turning the attention towards the effect of aging, it has been shown above that the $\Delta{u}/p-\varepsilon$ relationship does not change with aging, whereas the $\sigma_1'/\sigma_3'-\varepsilon$ curves show a steeper rise at small strains. Consequently, the $(\sigma_1-\sigma_3)-\varepsilon$ curves will rise more steeply and reach maximum values at smaller strains for longer aging periods. A decrease of failure strain with aging will also mean a corresponding smaller value of excess pore pressure at failure. For example, when the aging period for the CAU tests was increased from 3 days to 4 months the maximum value of $\sigma_1-\sigma_3$ changed only very little, about 6$\%$, whereas the pore pressure at failure was reduced by about 50$\%$. This explains why the A value is reduced from 0.88 to 0.41 in spite of the fact that the excess pore-pressure/strain relationship is not influenced by aging. Again, as the strain at failure decreases, the $\sigma_1'/\sigma_3'$ value at $(\sigma_1-\sigma_3)_{\max}$ is reduced, and for the samples which were consolidated for a period of 4 months the degree of mobilization at failure was as low as 77-78$\%$.

    \switchcolumn

    将注意力转移到老化的影响上,上面已经表明,$\Delta{u}/p-\varepsilon$关系不会随老化而改变,而$\sigma_1'/\sigma_3'-\varepsilon$曲线显示了在小应变下的陡峭上升。 因此,$(\sigma_1-\sigma_3)-\varepsilon$曲线将更陡峭地上升,并在较小的应变下达到更长的老化时间并达到最大值。 失效应变随老化的降低也将意味着失效时过大孔隙压力的相应较小值。 例如,当CAU试验的老化时间从3天增加到4个月时,$\sigma_1-\sigma_3$的最大值变化很小,约为6$\%$,而失效时的孔隙压力降低了约50$\%$。 这解释了为什么尽管过量的孔压/应变关系不受老化的影响,但$A$值仍从0.88降低到0.41。 同样,随着破坏应变的降低,$(\sigma_1-\sigma_3)_{\max}$处的$\sigma_1'/\sigma_3'$值也降低了,对于固结了4个月的样品,破坏时的调动程度低至77-78$\%$  。

\end{paracol}